\section{Deadlockerkennung allgemein}
\begin{itemize}
  \item Unterschiede zwischen statischer und dynamischer Deadlockerkennung
\end{itemize}

\section{Evaluierung von Methoden zur dynamischen Deadlockerkennung}
Methoden zur dynamischen Deadlockerkennung laufen in drei Schritten ab.
\begin{enumerate}
  \item Erstellung einer Trace-Datei
  \item Erstellung eines Graphens basierend auf den Informationen aus der
  Trace-Datei
  \item Finden von potenziellen Deadlocks durch das Identifizieren von Zyklen
  innerhalb des Graphens
\end{enumerate}

Eine Trace-Datei enthält einen \textit{execution trace} des ausführenden
Programme. Ein \textit{execution trace} ist eine Abfolge von Events. Ein Event
\textit{e\textsubscript{i}} wird durch eine der folgenden Methoden definiert:
starten eines Threads, Inbesitznahme eines Lockobjekts und Freigabe eines
Lockobjekts. Das Starten eines neues Threads ist definiert durch:
\begin{quote}
\texttt{s(Programmstelle, ausführender Thread, Name des neuen Threads)}
\end{quote}
Zum Beispiel bedeutet \texttt{s(2,main,T1)}, dass an der Programmstelle
\textit{2} der Thread \textit{main} den Thread \textit{T1} gestartet hat. 
Die Inbesitznahme eines Lockobjekts ist definiert durch:
\begin{quote}
\texttt{l(Programmstelle, ausführender Thread, Name des Lockobjekts)}
\end{quote}
Zum Beispiel bedeutet \texttt{l(24,T1,L3)}, dass
an der Programmstelle \textit{24} hat der Thread \textit{T1} das Lockobjekt
\textit{L3} in Besitz genommen. Die Freigabe eines Lockobjekts ist definiert
durch:
\begin{quote}
\texttt{u(Programmstelle, ausführender Thread, Name des Lockobjekts)}
\end{quote}
Zum Beispiel bedeutet \texttt{u(30, T1, L3)}, dass an der Programmstelle
\textit{30} der Thread \textit{T1} das Lockobjekt \textit{L3} freigegeben hat.

Die Menge aller während der Laufzeit des Programmes aufgetretenen Events
definieren einen möglichen \textit{execution trace} des Programmes.
Programme welche mit mehreren Threads arbeiten, haben keinen deterministischen
\textit{execution trace}. Jede Ausführung eines solchen Programmen kann zu
unterschiedlichen \textit{execution traces} führen. 

Im zweiten Schritt wird aus dem vorher erzeugten \textit{execution trace} ein
Lockgraph erstellt. Ein Lockgraph ist definiert durch:
\begin{quote}
\textit{LG = (L,R)}
\end{quote}
\textit{L} ist die Menge aller Lockobjekte im \textit{execution trace} und
\textit{R} die Menge aller Lockpaare. Ein Lockpaar ist definiert durch das Tupel
\textit{(l\textsubscript{1}, l\textsubscript{2})} für das gilt: Es existiert ein
Thread, welche das Lockobjekt \textit{l\textsubscript{1}} besitzt, während er
den Lock \textit{l\textsubscript{2}} anfordert.

In \cref{fig:CodeExample} ist ein Beispielprogramm in der Programmiersprache
OpenPEARL dargestellt.



\section{Ergebnis}\label{Ergebnis}
\begin{itemize}
  \item Auswahl MagicLock\cite{MagicLock}
\end{itemize}
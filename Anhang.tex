\startcontents
\chapter{OpenPEARL}
\thispagestyle{fancy}
\setminted{fontsize=\small,baselinestretch=1,breaklines,breakanywhere}

\inputminted{cpp}{./OpenPEARL/Semaphore.cc}
\begingroup
    \captionsetup{type=listing}
    \captionof{listing}{Angepasste Semaphore.cc Implementierung der OpenPEARL Laufzeitumgebung} 
\endgroup

\inputminted{bash}{./OpenPEARL/Files.common}
\begingroup
    \captionsetup{type=listing}
    \captionof{listing}{Angepasste Files.common der OpenPEARL Laufzeitumgebung} 
\endgroup

\stopcontents

\startcontents
\chapter{C++}
\thispagestyle{fancy}
\setminted{fontsize=\small,baselinestretch=1,breaklines,breakanywhere}

\inputminted{cpp}{./cpp/LockTraceEntry.h}
\begingroup
    \captionsetup{type=listing}
    \captionof{listing}{Header-Datei der Repräsentation eines Logeintrags} 
\endgroup

\inputminted{cpp}{./cpp/LockTraceEntry.cc}
\begingroup
    \captionsetup{type=listing}
    \captionof{listing}{Implementierung der Repräsentation eines Logeintrags} 
\endgroup

\inputminted{cpp}{./cpp/LockTraceEntryFormatter.h}
\begingroup
    \captionsetup{type=listing}
    \captionof{listing}{Header-Datei des Formatieres für Logeinträge} 
\endgroup

\inputminted{cpp}{./cpp/LockTraceEntryFormatter.cc}
\begingroup
    \captionsetup{type=listing}
    \captionof{listing}{Implementierung des Formatieres für Logeinträge} 
\endgroup

\inputminted{cpp}{./cpp/LockTraceEntryType.h}
\begingroup
    \captionsetup{type=listing}
    \captionof{listing}{Header-Datei der Enumeration für den Typ eines Logeintrags} 
\endgroup

\inputminted{cpp}{./cpp/LockTracer.h}
\begingroup
    \captionsetup{type=listing}
    \captionof{listing}{Header-Datei des Log-Tracers} 
\endgroup

\inputminted{cpp}{./cpp/LockTracer.cc}
\begingroup
    \captionsetup{type=listing}
    \captionof{listing}{Implementierung des Log-Tracers} 
\endgroup

\stopcontents

\startcontents
\chapter{Python}
\thispagestyle{fancy}
\setminted{fontsize=\small,baselinestretch=1,breaklines,breakanywhere}

\inputminted{python}{./Python/traceFileReader.py}
\begingroup
    \captionsetup{type=listing}
    \captionof{listing}{traceFileReader.py: Implementierung des Trace File Readers} 
\endgroup

\inputminted{python}{./Python/generateTimeline.py}
\begingroup
    \captionsetup{type=listing}
    \captionof{listing}{generateTimeline.py: Skript zur chronologischen Darstellung der Lock-Ereignisse} 
\endgroup

\inputminted{python}{./Python/magiclockLib/magicLockTypes.py}
\begingroup
    \captionsetup{type=listing}
    \captionof{listing}{magiclockLib/magicLockTypes.py: Sammlung von Klassen die von der Magiclock Implementierung verwendet werden} 
\endgroup

\inputminted{python}{./Python/magiclockLib/lockReduction.py}
\begingroup
    \captionsetup{type=listing}
    \captionof{listing}{magiclockLib/lockReduction.py: Implementierung des Magiclock-Algorithmus zur Reduzierung von Locks} 
\endgroup

\inputminted{python}{./Python/magiclockLib/cycleDetection.py}
\begingroup
    \captionsetup{type=listing}
    \captionof{listing}{magiclockLib/cycleDetection.py: Implementierung des Magiclock-Algorithmus zur Zyklenerkennung} 
\endgroup

\inputminted{python}{./Python/magiclockLib/magiclock.py}
\begingroup
    \captionsetup{type=listing}
    \captionof{listing}{magiclockLib/magiclock.py: Implementierung des Magiclock-Algorithmus} 
\endgroup

\inputminted{python}{./Python/generateDeadlockGraph.py}
\begingroup
    \captionsetup{type=listing}
    \captionof{listing}{generateDeadlockGraph.py: Skript zur Erkennung und Darstellung von potentiellen Deadlocks} 
\endgroup

\inputminted{python}{./Python/benchmark_cpu.py}
\begingroup
    \captionsetup{type=listing}
    \captionof{listing}{benchmark\_cpu.py: Skript zur Messung der CPU-Laufzeit einer OpenPEARL Anwendung} 
\endgroup

\inputminted{python}{./Python/benchmark_memory.py}
\begingroup
    \captionsetup{type=listing}
    \captionof{listing}{benchmark\_memory.py: Skript zur Messung der Speicherauslastung einer OpenPEARL Anwendung} 
\endgroup

\stopcontents
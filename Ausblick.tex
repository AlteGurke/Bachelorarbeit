Ziel dieser Arbeit war es, eine Unterstützung für Entwickler zu schaffen, um die
fehlerhafte Verwendung von Synchronisationsmitteln in der
Echtzeit-Programmiersprache PEARL zu erkennen.

Um dies zu erreichen wurden Verfahren zur Erkennung von Deadlocks vorgestellt
und implementiert. Es wurde ein Konzept für die OpenPEARL-Laufzeitumgebung
vorgestellt, um die benötigten Informationen zur Visualisierung von
Synchronisationsmitteln und zur Erkennung von Deadlocks, in eine Trace-Datei zu
speichern. Zusätzlich wurden zwei Anwendungen implementiert. Eine Anwendung zur
Visualisierung der chronologischen Verwendung von Synchronisationsmitteln und
eine zweite zur Erkennung und Darstellung von potenziellen Deadlocks.

Die entwickelten Funktionalitäten konnten erfolgreich getestet werden.
Zusätzlich konnten Empfehlungen zur Konfiguration der Trace""-Funktionalität in
der OpenPEARL""-Laufzeitumgebung gemacht werden. Diese Empfehlungen können
verwendet werden, um einen guten Kompromiss zwischen Laufzeit und
Speicherauslastung auf dem jeweiligen Zielsystem zu erreichen.

Bei der Betrachtung der Synchronisationsmittel in PEARL wurden in dieser Arbeit
nur \textrm{SEMA}-Variablen einbezogen. Die erstellten Hilfsmittel für
Entwickler sind demnach limitiert und können zum Beispiel keine Deadlocks
erkennen, in denen \textrm{BOLT}-Variablen involviert sind. Die erzielten
Ergebnisse liefern eine gute Grundlage, um die Unterstützung für
\textrm{BOLT}-Variablen hinzuzufügen.

Die Integration der Trace-Funktionalität in die OpenPEARL-Laufzeitumgebung ist
derzeit kein Bestandteil des OpenPEARL-Projekts. Um dies zu erreichen müssen
noch Anpassungen gemacht werden. Die Robustheit der Funktionalität muss
überprüft und sichergestellt werden, damit Anwendungen zur Laufzeit kein
ungewolltes Verhalten aufweisen. Zusätzlich sollte der Weg über eine Trace-Datei
verbessert werden. Eine effizientere Lösung wäre es, die benötigten
Informationen anstatt in eine Datei direkt über Interprozesskommunikation an
einen zweiten Prozess zu übermitteln. Dieser kann die Informationen asynchron in
eine Datenbank ablegen. Die in dieser Arbeit vorgestellten Implementierungen
können mit den Informationen in der Datenbank die benötigten Analysen
durchführen und dem Entwickler die Ergebnisse über eine Webschnittstelle zur
Verfügung stellen.
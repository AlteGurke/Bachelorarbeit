\section{Offene Punkte}
\label{section:OffenePunkte}
In OpenPEARL gibt es neben den \emph{SEMA} Variablen noch \emph{BOLT} Variablen,
welche von der aktuellen Implementierung nicht beachtet werden. Um alle
Synchronisierungsprobleme in OpenPEARL Anwendungen aufzeigen zu können, muss die
Unterstützung für \emph{BOLT} Variablen hinzugefügt werden.

Die derzeitige Integration in die OpenPEARL Laufzeitumgebung erfüllt noch nicht
alle Kriterien in Bezug auf Softwarequalität, gerade im Bereich Robustheit. Aus
diesem Grund ist die Integration noch kein Bestandteil des offiziellen OpenPEARL
Repository.

Zusätzlich ist durch die Verwendung der lock freien Warteschlange keine
Reihenfolge garantiert. Die verwendeten Zeitstempel sind abhängig von der
jeweiligen Implementierung und können von der Genauigkeit her stark variieren.
Dadurch kann es passieren, dass mehrere Lockereignisse mit gleichen Zeitstempel
erzeugt werden. Wenn bei diesen Ereignissen die Reihenfolge nicht garantiert
ist, dann kann dies die gesamte Analyse verfälschen. Eine mögliche Lösung ist
es, den Lockereignissen zusätzlich zum Zeitstempel eine Sequenznummer
mitzugeben. Dadurch wird die Speicherauslastung zwar leicht erhöht, die
Reihenfolge kann beim Auslesen aber sichergestellt werden.

\section{Weiterentwicklung}
\label{section:Weiterentwicklung}
Der nächste Entwicklungsschritt ist die Erkennung von potenziellen Deadlocks in
Echtzeit während der Ausführung einer OpenPEARL Anwendung. Dazu muss die
\texttt{LockTracer}-Implementierung aus \cref{lst:LockTracer} erweitert werden.
Anstatt die Ereignisse in eine Trace-Datei zu schreiben, müssen diese direkt
über Interprozesskommunikation von der OpenPEARL Laufzeitumgebung an einen
zweiten Prozess übertragen werden. Dieser kann dann die Python-Implementierungen
aus \cref{section:Implementierung:Analyse Programm} und
\cref{section:Implementierung:Visualisierung von potenziellen Deadlocks}
aufrufen und die Analysen direkt durchführen. Die Visualisierung muss dann in
regelmäßigen Intervallen aktualisiert werden, um dem Benutzer die Änderungen
anzuzeigen.
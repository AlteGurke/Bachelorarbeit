\section{Offene Punkte}
\label{section:OffenePunkte}
In OpenPEARL gibt es neben den \textrm{SEMA}-Variablen noch
\textrm{BOLT}-Variablen, die von der aktuellen Implementierung nicht beachtet
werden. Um alle Synchronisierungsprobleme in OpenPEARL-Anwendungen aufzeigen zu
können, muss die Unterstützung für \textrm{BOLT}-Variablen hinzugefügt werden.

Die derzeitige Integration in die OpenPEARL-Laufzeitumgebung erfüllt noch nicht
alle Kriterien in Bezug auf Softwarequalität, gerade im Bereich Robustheit. Aus
diesem Grund ist die Integration noch kein Bestandteil des offiziellen OpenPEARL
Repository.

\section{Weiterentwicklung}
\label{section:Weiterentwicklung}
Der nächste Entwicklungsschritt ist die Erkennung von potenziellen Deadlocks in
Echtzeit während der Ausführung einer OpenPEARL-Anwendung. Dazu muss die
\texttt{LockTracer}-Implementierung aus \cref{lst:LockTracer} erweitert werden.
Anstatt die Ereignisse in eine Trace-Datei zu schreiben, müssen diese direkt
über Interprozesskommunikation von der OpenPEARL-Laufzeitumgebung an einen
zweiten Prozess übertragen werden. Dieser kann dann die Python-Implementierungen
aus \cref{section:Implementierung:Analyse-Programm} und
\cref{section:Implementierung:Visualisierung von potenziellen Deadlocks}
aufrufen und die Analysen direkt durchführen. Die Visualisierung muss dann in
regelmäßigen Intervallen aktualisiert werden, um dem Benutzer die Änderungen
anzuzeigen.
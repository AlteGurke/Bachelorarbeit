Ziel dieser Arbeit war es, eine Unterstützung für Entwickler zu schaffen, um die
fehlerhafte Verwendung von Synchronisationsmitteln in der
Echtzeit-Programmiersprache PEARL zu erkennen.

Um dies zu erreichen wurden Verfahren zur Erkennung von Deadlocks vorgestellt
und implementiert. Es wurde ein Konzept für die OpenPEARL-Laufzeitumgebung
vorgestellt, um die benötigten Informationen zur Visualisierung von
Synchronisationsmitteln und zur Erkennung von Deadlocks, in eine Trace-Datei zu
speichern. Zusätzlich wurden zwei Anwendungen implementiert. Eine Anwendung zur
Visualisierung der chronologischen Verwendung von Synchronisationsmitteln und
eine zweite zur Erkennung und Darstellung von potenziellen Deadlocks.

Die entwickelten Funktionalitäten konnten erfolgreich getestet werden.
Zusätzlich konnten Empfehlungen zur Konfiguration der Trace""-Funktionalität in
der OpenPEARL""-Laufzeitumgebung gemacht werden. Diese Empfehlungen können
verwendet werden, um einen guten Kompromiss zwischen Laufzeit und
Speicherauslastung auf dem jeweiligen Zielsystem zu erreichen.

Das Ergebnis dieser Arbeit ist eine Unterstützung für Entwickler von
PEARL-Programmen in Bezug auf Nebenläufigkeitsprobleme. Entwickler können sich
die chronologische Belegung von Synchronisationsmitteln und potentielle
Deadlocks darstellen lassen. Mit diesen Informationen ist es den Entwicklern
möglich die Probleme zu lokalisieren. Zusätzlich kann zum Beispiel das Entfernen
eines potentiellen Deadlocks überprüft werden, in dem ein zweiter Testlauf
durchgeführt und auf potentielle Deadlocks untersucht wird.

Bei der Betrachtung der Synchronisationsmittel in PEARL wurden in dieser Arbeit
nur \textrm{SEMA}-Variablen einbezogen. Die erstellten Hilfsmittel für
Entwickler sind demnach limitiert und können zum Beispiel keine Deadlocks
erkennen, in denen \textrm{BOLT}-Variablen involviert sind. Die erzielten
Ergebnisse liefern eine gute Grundlage, um die Unterstützung für
\textrm{BOLT}-Variablen hinzuzufügen.

Die Integration der Trace-Funktionalität in die OpenPEARL-Laufzeitumgebung ist
derzeit kein Bestandteil des OpenPEARL-Projekts. Um dies zu erreichen müssen
noch Anpassungen gemacht werden. Die Robustheit der Funktionalität muss
überprüft und sichergestellt werden, damit Anwendungen zur Laufzeit kein
ungewolltes Verhalten aufweisen.

\section{Ausblick}
Bei der Implementierung der Analyse-Anwendungen wurde darauf geachtet, dass die
Ausführung der PEARL-Anwendung so wenig wie möglich beeinflusst wird. Deswegen
werden die Informationen für die Analysen zwar zur Laufzeit erstellt, die
Analysen selbst werden aber nachfolgend in externen Prozessen durchgeführt. Die
Lockereignisse werden von den Analysen in einer chronologischen Reihenfolge
benötigt. Diese Reihenfolge wird von der Implementierung garantiert. Dadurch ist
es möglich die Analysen auch direkt zur Laufzeit auszuführen, zum Beispiel genau
dann wenn ein neues Lockereignis auftritt.

Der Vorteil einer Analyse direkt zur Laufzeit liegt darin, dass Entwickler die
Informationen direkt dargestellt bekommen und schneller Fehler erkennen und
darauf reagieren können. Ein möglicher Nachteil kann der deutlich erhöhte Bedarf
an Prozessorleistung und Hauptspeicher sein. Das Sammeln der Informationen
beeinflusst die Laufzeit nur geringfügig und führt nur zu einer moderaten
Erhöhung der Speicherauslastung der Anwendung, wenn die Puffergröße entsprechend
gewählt wurde. 

Würden die Analysen direkt zur Laufzeit durchgeführt werden, müssten immer alle
Lockereignisse in den Hauptspeicher geladen werden. Zusätzlich werden während
der Analysen weitere Datenstrukturen erzeugt, welche den Speicherbedarf weiter
erhöhen würden. Die Speicherauslastung wäre damit garantiert höher als die
Speicherauslastung der aktuellen Implementierung mit maximaler Puffergröße.

Das Erstellen und Puffern der Lockereignisse erhöht die Laufzeit nur marginal.
Den größten Einfluss auf die Laufzeit hat das Schreiben der Lockereignisse in
die Trace-Datei. Bei der Analyse zur Laufzeit könnte dieser Schritt entfallen,
wodurch sich die Laufzeit verringern würde. Die Analysen durchlaufen immer alle
Lockereignisse. Beim Durchführen der Analysen beim Auftreten eines
Lockereignisses, würde sich die Laufzeit der Analyse jedes Mal erhöhen. Dadurch
würde die Laufzeit der PEARL-Anwendung für jedes Auftreten eines Lockereignisses
immer stärker beeinflusst werden. Bei Zielsystemen mit langsamen Speicher
könnten die Analysen, bis zu einer gewissen Anzahl an Lockereignissen, die
Laufzeit weniger stark beeinflussen als das Speichern der Trace-Datei.

Die in dieser Arbeit vorgestellten Analysen und deren Implementierungen können
für die Analyse direkt zur Laufzeit verwendet werden. Die Ergebnisse dieser
Arbeit können somit als Ausgangspunkt für die Weiterentwicklung verwendet
werden.
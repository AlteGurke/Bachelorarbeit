\section{OpenPEARL}
\begin{itemize}
    \item Kurze Beschreibung der Programmiersprache OpenPEARL
    \item Aufzeigen der Möglichkeiten von Synchronisationmitteln in OpenPEARL
\end{itemize}

\section{Erzeugung der Trace-Datei}
\begin{itemize}
    \item Definition der notwendigen Informationen der Trace-Datei
    \item Aufzeigen der Herausforderungen in Bezug auf Performance und
    Speicherauslastung
  \item Anforderungen definieren:
  \begin{itemize}
    \item Laufzeit des Programmes soll sich um maximal x\% erhöhen
    \item Speicherauslastung des Programmes soll sich um maximal x\% erhöhen
  \end{itemize}
\end{itemize}

\section{Analysieren der Trace-Datei}\label{Analysieren der Trace-Datei}
\begin{itemize}
  \item Externes Programm geschrieben in Java
  \item Anforderungen definieren:
  \begin{itemize}
    \item Darstellung der Trace-Datei (welcher Thread hat welches
    Synchronisationmittel wann genommen und wieder freigegeben) 
  \end{itemize}
\end{itemize}

\section{Erweiterung: Potenzielle Deadlocks}
\begin{itemize}
  \item Programm aus \cref{Analysieren der Trace-Datei} wird erweitert
  \item Es sollen potenzielle Deadlocks mit Hilfe des aus \cref{Ergebnis}
  ausgewählten Verfahrens bestimmt werden
  \item Potenzielle Deadlocks sollen als gerichtete Graphen visualisiert werden
\end{itemize}
\section{Trace Funktion}
\label{section:Implementierung:Trace Funktion}
Aus dem in \cref{fig:LockTrace_Design} vorgestellten UML-Diagramm werden drei
Klassen in C++ implementiert. \texttt{LockTraceEntryType} wird als \texttt{enum}
implementiert, die Klasse \texttt{LockTraceEntry} als einfache Datenklasse mit
Get-Methoden für jedes Attribut und die Klasse \texttt{LogTraceEntryFormatter}
als Singleton.


\section{Analyse Programm}
\label{section:Implementierung:Analyse Programm}
\begin{itemize}
  \item Vorstellung der Implementierung des Analyse Programms in Java
  \item Grafiken/Screenshots mit Analyse Beispielen
\end{itemize}

\section{Visualisierung von potenziellen Deadlocks}
\label{section:Implementierung:Visualisierung von potenziellen Deadlocks}
\begin{itemize}
  \item Vorstellung der Implementierung des Algorithmus zur Erkennung von
  potenziellen Deadlocks
  MagicLock klassifiziert dazu jedes Lock-Objekt in eine der folgenden Mengen:
  \begin{enumerate}
    \item \textbf{Independent-set} = $\{m \mid m \in Locks, indegree(m) = 0 \land outdegree(m) = 0\}$
    \item \textbf{Intermediate-set} = $\{m \mid m \in Locks, (indegree(m) = 0 \lor outdegree(m) = 0) \land \lnot (indegree(m) = 0 \land outdegree(m) = 0)\}$
    \item \textbf{Inner-set} = $\{m \mid m \in Locks, (\exists (t,m,L) \in D, \forall n \in L, n \in \text{Intermediate-set} \cup \text{Inner-set}) \lor (\exists (t,n,L) \in D, m \in L \land n \in \text{Intermediate-set} \cup \text{Inner-set})\}$
    \item \textbf{Cyclic-set} = $\{m \mid m \in Locks, m \notin \text{Independent-set} \cup \text{Intermediate-set} \cup \text{Inner-set}\}$
  \end{enumerate}
  \item Grafiken/Screenshots mit Analyse Beispielen
\end{itemize}
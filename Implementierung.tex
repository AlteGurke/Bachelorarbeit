\section{Trace Funktion}
\label{section:Implementierung:Trace Funktion}
Aus dem in \cref{fig:LockTrace_Design} vorgestellten UML-Diagramm werden drei
Klassen und eine Enumeration in C++ implementiert.
\texttt{Lock\-Trace\-Entry\-Type} wird als \texttt{enum} mit den Werten
\texttt{Lock} und \texttt{Unlock} implementiert. Die Klasse
\texttt{LockTraceEntry} wird als einfache Datenklasse mit Get-Methoden für jedes
Attribut implementiert. Die beiden Klassen \texttt{Log\-Trace\-Entry\-Formatter}
und \texttt{LogTracer} werden als Singleton implementiert. Die Implementierung
für die Berechnung des genauen Zeitpunkts ist in
\cref{lst:LockTraceEntryFormatter_FormatLogTraceEntry} dargestellt.
\begin{listing}[ht]
  \inputminted[frame=lines,linenos,firstline=31,lastline=33]{cpp}{./cpp/LockTraceEntryFormatter.cc}
  \caption{Auszug aus LockTraceEntryFormatter.cc: Berechnung des Zeitpunkts}
  \label{lst:LockTraceEntryFormatter_FormatLogTraceEntry}   
\end{listing}
Der \texttt{chrono::time\_point} wird in Mikrosekunden seit dem 01.01.1970
umgerechnet. Mikrosekunden wurden als Genauigkeit gewählt, weil Millisekunden
nicht hochauflösend genug sind, um die chronologische Verwendung von
Lockobjekten darzustellen. Eine höhere Auflösung als Mikrosekunden würde die
Darstellung hingegen zu unübersichtlich machen. In \cref{lst:LockTracer} ist ein
Auszug aus der Implementierung der Klasse \texttt{LockTracer} dargestellt.
\begin{listing}[ht]
  \inputminted[frame=lines,linenos,firstline=11,lastline=58]{cpp}{./cpp/LockTracer.cc}
  \caption{LockTracer.cc: Auszug aus der Implementierung des LockTracers}
  \label{lst:LockTracer}
\end{listing}
Die \texttt{GetInstance()} Methode stellt sicher, dass nur eine Instanz der
Klasse zur Laufzeit existiert\footnote{Ab C++ Version 11}. Im Konstruktor wird
einmalig geprüft, ob die notwendigen Umgebungsvariablen existieren und korrekte
Werte aufweisen. Wenn alle Variablen vorhanden und die Variable
\texttt{Name\-Of\-Environment\-Variable\-Enabled} den Wert \texttt{true} hat,
wird die Trace-Funktionalität aktiviert. In der Methode
\texttt{Add(Lock\-Trace\-Entry\& entry)} wird zuerst geprüft, ob die
Trace-Funktionalität aktiviert ist. Ist dies nicht der Fall wird die Methode
sofort beendet, dadurch wird die Laufzeit der OpenPEARL Anwendung bei
deaktivierter Trace-Funktionalität nicht negativ beeinflusst\footnote{Der
einmalige Aufruf des Konstruktors, sowie der Aufruf der Add Methode und das
Prüfen eines booleschen Wertes werden hier ignoriert.}. Es wird sichergestellt,
dass beim Beenden der Anwendung alle noch vorhanden Lockereignisse in die
Trace-Datei geschrieben werden. Dies erfolgt über den Destruktor der Klasse in
Zeile 52. Die Integration in die OpenPEARL Laufzeitumgebung erfolgt durch das
Hinzufügen der benötigten Dateien in die \emph{Files.common} Datei. Dazu wird
die Zeile 86 um \cref{lst:OpenPEARLFilesCommon} erweitert.
\begin{listing}[ht]
  \inputminted[frame=lines,linenos,firstline=86,lastline=88]{bash}{./OpenPEARL/Files.common}
  \caption{Files.common: Auszug aus der Auflistung der zu kompilierenden Dateien}
  \label{lst:OpenPEARLFilesCommon}
\end{listing}
Im letzten Schritt wird die \emph{Semaphore.cc} Implementierung angepasst. Für
jede Erhöhung auf eins eines Semaphore-Objekts wird die Methode
\texttt{TraceUnlock}, für jede Verringerung auf null die Methode
\texttt{TraceLock} aus \cref{lst:OpenPEARLSemaphore} aufgerufen.
\begin{listing}[ht]
  \inputminted[frame=lines,linenos,firstline=93,lastline=111]{cpp}{./OpenPEARL/Semaphore.cc}
  \caption{Semaphore.cc: Auszug aus der Semaphore Implementierung in der OpenPEARL Laufzeitumgebung}
  \label{lst:OpenPEARLSemaphore}
\end{listing}

\section{Analyse Programm}
\label{section:Implementierung:Analyse Programm}
Die Implementierung für die Analyse und die chronologische Darstellung der
Verwendung von \emph{SEMA} Objekten wird in Python\footnote{Python Version
3.7.3.} durchgeführt. Im ersten Schritt wird die Trace-Datei aus
\cref{section:Erzeugung der Trace-Datei} ausgelesen. Die einzelnen
Lockereignisse werden von der Klasse
\texttt{Lock\-Action} repräsentiert, welche in \cref{lst:TraceFileReader}
dargestellt ist.
\begin{listing}[ht]
  \inputminted[frame=lines,linenos,firstline=1,lastline=10]{python}{./Python/traceFileReader.py}
  \caption{traceFileReader.py: Auszug aus der Implementierung des Trace-Datei Parsers}
  \label{lst:TraceFileReader}
\end{listing}
Anschließend wird aus den ausgelesenen Lockereignissen ein Graph erstellt. Dazu
wird die Python-Bibliothek Matplotlib\footnote{Matplotlib Version 3.1.3.}
verwendet. In \cref{lst:GenerateTimeline_CreateGraphValues} ist die Erzeugung
des Graphen dargestellt.
\begin{listing}[ht]
  \inputminted[frame=lines,linenos,firstline=16,lastline=38]{python}{./Python/generateTimeline.py}
  \caption{generateTimeline.py: Auszug aus der Bestimmung der einzelnen Werte für den Graphen}
  \label{lst:GenerateTimeline_CreateGraphValues}
\end{listing}
In den Zeilen 23 bis 25 werden die Threads und Zeitstempel in Hashsets
gespeichert und anschließend in den Zeilen 27 und 28 sortiert. In den Zeilen 30
bis 38 werden die Werte für die einzelnen Lockereignisse bestimmt. Um die
Threads auf der Ordinate abzubilden, wird in Zeile 36 der Index des jeweiligen
Threads im sortierten Hashset verwendet. Dadurch entspricht jeder ganzzahlige
Wert auf der Ordinate einem Thread. Falls es mehrere Einträge für eine Thread
mit dem gleichen Zeitstempel gibt, wird in den Zeilen 33 und 34 ein Offset
berechnet und in der Zeile 36 hinzugefügt, um eine Überlappung zu verhindern.
Für den Wert auf der Abszisse wird das Delta der Zeitstempel zwischen dem
jeweiligen Lockereignis und dem niedrigsten Zeitstempel in der Zeile 31
berechnet. In Zeile 37 wird die Farbe des Lockereignisses und in Zeile 38 die
Beschriftung festgelegt. Aus den einzelnen Werten wird in
\cref{lst:GenerateTimeline_PlotGraph} der Graph mit Matplotlib erstellt und kann
dann mit dem Befehl \texttt{plt.show()} dargestellt werden.
\begin{listing}[ht]
  \inputminted[frame=lines,linenos,firstline=42,lastline=44]{python}{./Python/generateTimeline.py}
  \caption{generateTimeline.py: Auszug aus der Erzeugung des Graphen}
  \label{lst:GenerateTimeline_PlotGraph}
\end{listing}

\section{Visualisierung von potenziellen Deadlocks}
\label{section:Implementierung:Visualisierung von potenziellen Deadlocks}
Im ersten Schritt wird der MagicLock-Algorithmus in Python\footnote{Python
Version 3.7.3.} implementiert. Das Auslesen der Trace-Datei wurde bereits in
\cref{section:Implementierung:Analyse Programm} durchgeführt und wird hier
wiederverwendet. Nach dem Auslesen werden die Lockereignisse in eine
Lock-Dependency-Relation\footnote{Siehe \cref{section:MagicLock}} überführt. Die
Klassen sind in \cref{lst:MagiclockTypes_LockDependency} und
\cref{lst:MagiclockTypes_LockDependencyRelation} dargestellt.
\begin{listing}[ht]
  \inputminted[frame=lines,linenos,firstline=45,lastline=49]{python}{./Python/magiclockLib/magiclockTypes.py}
  \caption{magiclockLib/magiclockTypes.py: Repräsentation einer \emph{Lock Dependency} aus Magiclock\autocite[3]{MagicLock}}
  \label{lst:MagiclockTypes_LockDependency}
\end{listing}
\begin{listing}[ht]
  \inputminted[frame=lines,linenos,firstline=67,lastline=77]{python}{./Python/magiclockLib/magiclockTypes.py}
  \caption{magiclockLib/magiclockTypes.py: Repräsentation einer \emph{Lock Dependency Relation} aus Magiclock\autocite[3]{MagicLock}}
  \label{lst:MagiclockTypes_LockDependencyRelation}
\end{listing}
Eine \texttt{LockDependencyRelation} enthält ein Hashset mit allen Lockobjekten,
eine Liste aller Threads und eine Liste der einzelnen \texttt{LockDependency}
Objekten. Für die Threads wurde eine Liste gewählt, damit die Ergebnisse
deterministisch ausfallen. Die Liste der Threads wird später in einer Schleife
durchlaufen. Dies führt bei einem Hashset, auf Grund der zufälligen Sortierung,
zu nicht deterministischen Ergebnissen.

Im zweiten Schritt wird die \texttt{LockDependencyRelation} reduziert. Dazu
werden die einzelnen Lockobjekte klassifiziert durch die Zuweisung in eine der
folgenden Mengen\autocite[4]{MagicLock}.
\begin{enumerate}
  \item \textbf{Independent-set} = $\{m \mid m \in Locks, indegree(m) = 0 \land
  outdegree(m) = 0\}$
  \item \textbf{Intermediate-set} = $\{m \mid m \in Locks, (indegree(m) = 0 \lor
  outdegree(m) = 0) \land \lnot (indegree(m) = 0 \land outdegree(m) = 0)\}$
  \item \textbf{Inner-set} = $\{m \mid m \in Locks, (\exists (t,m,L) \in D,
  \forall n \in L, n \in \text{Intermediate-set} \cup \text{Inner-set}) \lor
  (\exists (t,n,L) \in D, m \in L \land n \in \text{Intermediate-set} \cup
  \text{Inner-set})\}$
  \item \textbf{Cyclic-set} = $\{m \mid m \in Locks, m \notin
  \text{Independent-set} \cup \text{Intermediate-set} \cup \text{Inner-set}\}$
\end{enumerate}
Dazu werden die Algorithmen \emph{LockReduction(D)},
\emph{InitClassification(D)} und \emph{LockClassification(D)}
implementiert\footnote{Entspricht den Algorithmen 1, 2 und 3 aus
MagicLock\autocite[5]{MagicLock}}. Als erstes wird in
\cref{lst:LockReduction_InitClassification} mit \emph{InitClassification(D)}
eine Datenstruktur initialisiert.
\begin{listing}[ht]
  \inputminted[frame=lines,linenos,firstline=17,lastline=32]{python}{./Python/magiclockLib/lockReduction.py}
  \caption{magiclockLib/lockReduction.py: Implementierung des \emph{InitClassification(D)} Algorithmus aus Magiclock\autocite[5]{MagicLock}}
  \label{lst:LockReduction_InitClassification}
\end{listing}
Für jede \texttt{LockDependency} werden die eingehenden und ausgehenden Kanten
sowie der Mode bestimmt. Der Mode ist entweder der Name des Threads, welcher als
einziger Thread in der \texttt{LockDependencyRelation} Zugriffe auf das
Lockobjekt ausführt, oder -1. Die erstellte Datenstruktur ist in
\cref{lst:MagiclockTypes_InitClassification} dargestellt.
\begin{listing}[ht]
  \inputminted[frame=lines,linenos,firstline=3,lastline=8]{python}{./Python/magiclockLib/magiclockTypes.py}
  \caption{magiclockLib/magiclockTypes.py: Datenstruktur der \texttt{init\_Classification(D)} Methode}
  \label{lst:MagiclockTypes_InitClassification}
\end{listing}
Die Klassifizierung der Lockobjekte erfolgt in
\cref{lst:LockReduction_LockClassification}.
\begin{listing}[ht]
  \inputminted[frame=lines,linenos,firstline=35,lastline=77]{python}{./Python/magiclockLib/lockReduction.py}
  \caption{magiclockLib/lockReduction.py: Implementierung des \emph{LockClassification(D)} Algorithmus aus Magiclock\autocite[5]{MagicLock}}
  \label{lst:LockReduction_LockClassification}
\end{listing}
Alle Lockobjekte ohne eingehende und ausgehende Kanten werden in den Zeilen 39
und 40 in das \textbf{Independent-set} eingefügt. Diese Lockobjekte können
ignoriert werden, da ein potenzieller Deadlock mindestens eine ausgehende und
mindestens eine eingehende Kante besitzen muss. Lockobjekte ohne eingehende oder
ohne ausgehende Kanten werden in den Zeilen 42 bis 44 in das
\textbf{Intermediate-set} eingefügt. Zusätzlich werden diese auf den Stack
\texttt{s} gelegt. In den Zeilen 46 bis 69 wird die Klassifizierung für das
\textbf{Inner-set} durchgeführt. Das oberste Lockobjekt \texttt{m} vom Stack \texttt{s}
wird entfernt und überprüft. Dies wird wiederholt bis der Stack keine
Lockobjekte mehr enthält. Wenn das Lockobjekt \texttt{m} keine eingehenden
Kanten besitzt, werden alle anderen Lockobjekte mit eingehenden Kanten in den
Zeilen 49 bis 58 durchlaufen. Die eingehenden Kanten des Lockobjekts \texttt{n}
können dann um die Anzahl der Kanten von dem Lockobjekt \texttt{m} zu \texttt{n}
reduziert werden. Wenn das Lockobjekt \texttt{n} anschließend selbst keine
eingehenden Kanten mehr besitzt, wird es auf den Stack \texttt{s} gelegt und in
das \textbf{Intermediate-set} eingefügt. Anschließend werden in den Zeilen 57
und 58 die Kanten von dem Lockobjekt \texttt{m} zu \texttt{n} und alle
ausgehenden Kanten von \texttt{m} auf 0 gesetzt. Das Gleiche wird für
Lockobjekte ohne ausgehende Kanten in den Zeilen 59 bis 69 gemacht. Zuletzt
werden in den Zeilen 71 bis 75 alle Lockobjekte, welche in keiner der bisherigen
Mengen vorhanden sind, in das \textbf{Cyclic-set} eingefügt.

Der Algorithmus zur Reduzierung der \texttt{LockDependencyRelation} ist in
\cref{lst:LockReduction_LockReduction} dargestellt.
\begin{listing}[ht]
  \inputminted[frame=lines,linenos,firstline=90]{python}{./Python/magiclockLib/lockReduction.py}
  \caption{magiclockLib/lockReduction.py: Implementierung des \emph{LockReduction(D)} Algorithmus aus Magiclock\autocite[5]{MagicLock}}
  \label{lst:LockReduction_LockReduction}
\end{listing}
Als erstes wird die Klassifizierung der Lockobjekte in Zeile 91 durchgeführt.
Anschließend wird in den Zeilen 93 bis 103 das \textbf{Cyclic-set} durchlaufen
und alle Lockobjekte mit einem Mode ungleich -1 aus diesem entfernt. Diese
Lockobjekte können nicht Teil eines potenziellen Deadlocks sein, da sie nur von
einem einzigen Thread verwendet werden. In der Zeile 105 wird eine neue
\texttt{LockDependencyRelation} erzeugt, welche nur Lockobjekte aus dem
\textbf{Cyclic-set} enthält. Wenn sich diese Relation nicht von der
ursprünglichen unterscheidet ist die Lock Reduzierung abgeschlossen. Ansonsten
wird in Zeile 107 die Lock Reduzierung rekursiv mit der eben erzeugten
\texttt{LockDependencyRelation} erneut aufgerufen.

Nach der Reduzierung der Lockobjekte werden die vorhandenen Lockobjekte aus dem
\textbf{Cyclic-set} in disjunkte Mengen aufgeteilt. Die disjunkten Mengen haben
untereinander keine Kanten und sind daher unabhängige Teilgraphen. Die Suche
nach Zyklen erfolgt anschließend für jeden dieser Teilgraphen. In
\cref{lst:CycleDetection_DisjoinComponents} wird die Aufteilung in disjunkte
Mengen durchgeführt. 
\begin{listing}[ht]
  \inputminted[frame=lines,linenos,firstline=1,lastline=25]{python}{./Python/magiclockLib/cycleDetection.py}
  \caption{magiclockLib/cycleDetection.py: Implementierung des \emph{DisjointComponentsFinder(Cyclic-set)} Algorithmus aus Magiclock\autocite[8]{MagicLock}}
  \label{lst:CycleDetection_DisjoinComponents}
\end{listing}
In den Zeilen 19 bis 23 werden die einzelnen Lockobjekte aus dem
\textbf{Cyclic-set} durchlaufen und falls noch nicht betrachtet in der Methode
\texttt{visit\_Edges\_From} betrachtet. In der Methode \texttt{visit\_Edges\_From}
wird das Lockobjekt, falls noch nicht betrachtet, in die disjunkte Menge
übernommen. Zusätzlich werden in den Zeilen 6 bis 8 alle erreichbaren
Lockobjekte durchlaufen und ebenfalls in die disjunkte Menge aufgenommen.

Im letzten Schritt wird für jede disjunkte Menge eine Zyklensuche durchgeführt,
wobei jeder Zyklus einen potenziellen Deadlock repräsentiert. Dazu werden zuerst
in \cref{lst:CycleDetection_CycleDetection} in den Zeilen 94 bis 102 Partitionen
erstellt.
\begin{listing}[ht]
  \inputminted[frame=lines,linenos,firstline=78,lastline=108]{python}{./Python/magiclockLib/cycleDetection.py}
  \caption{magiclockLib/cycleDetection.py: Implementierung des \emph{CycleDetection(dc, D)} Algorithmus aus Magiclock\autocite[8]{MagicLock}}
  \label{lst:CycleDetection_CycleDetection}
\end{listing}
Es wird für jeden Thread \emph{t} eine Partition erstellt, welche die
\texttt{LockDependency} Objekte enthält, bei denen der ausführende Thread gleich
\emph{t} ist. Zusätzlich werden identische \texttt{LockDependency} Objekte
gruppiert. Ein \texttt{LockDependency} Objekt ist identisch mit einem anderen
\texttt{LockDependency} Objekt, wenn der ausführende Thread, das betroffene
Lockobjekt und die Menge der aktuell in Besitz befindlichen Lockobjekte
übereinstimmt. Anschließend wird in den Zeilen 105 bis 108 für jeden Thread die
jeweilige Partition durchlaufen. Jede \texttt{LockDependency} in der Partition
wird mit der Methode \texttt{DFS\_Traverse} in
\cref{lst:CycleDetection_DFSTraverse} nach Zyklen durchsucht.
\begin{listing}[ht]
  \inputminted[frame=lines,linenos,firstline=50,lastline=75]{python}{./Python/magiclockLib/cycleDetection.py}
  \caption{magiclockLib/cycleDetection.py: Implementierung des \emph{DFS\_Traverse(i, S, $\tau$)} Algorithmus aus Magiclock\autocite[8]{MagicLock}}
  \label{lst:CycleDetection_DFSTraverse}
\end{listing}
Zuerst wird in Zeile 61 eine Menge \emph{s} erzeugt, die anfangs nur die
aktuelle \texttt{LockDependencyRelation} enthält. In Zeile 62 werden alle höher
sortierten Threads durchlaufen, wobei bereits betrachtete Threads in den Zeilen
63 und 64 ignoriert werden. Anschließend werden in den Zeilen 65 bis 75 die
\texttt{LockDependency} Objekte in der Partition des höheren Threads
durchlaufen. Für jedes dieser Objekte wird in den Zeilen 68 und 69 geprüft, ob
die aktuelle Menge \emph{s} zusammen mit dieser \texttt{LockDependency} eine
Zyklische-Lock-Dependency-Chain\footnote{Siehe \cref{section:MagicLock}} bildet.
Ist dies nicht der Fall, wird die Methode \texttt{DFS\_Traverse} rekursiv für
die aktuell in der Iteration befindliche \texttt{LockDependency} aufgerufen.
Falls es sich um eine Zyklische-Lock-Dependency-Chain handelt wird für den
Zyklus und jeden äquivalenten Zyklus ein potenzieller Deadlock in den Zeilen 50
bis 57 gemeldet.
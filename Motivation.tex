\section{Motivation}

Bei der parallelen Programmierung ist die korrekte Verwendung von
Synchronisationsmitteln ein Problem. Während der Entwicklung sind
Nebenläufigkeitsprobleme für den Entwickler nur sehr schwer zu erkennen.
Zusätzlich unterstützen moderne Entwicklungsumgebungen und Compiler den
Entwickler nicht oder nur unzureichend bei diesen Problemen.

Eines dieser Probleme ist das Auftreten von \textit{Deadlocks}, welche in
\cref{section:Deadlockerkennung allgemein} beschrieben werden.
Das grundlegende Verfahren zur Identifizierung von potenziellen Deadlocks
ist in \cref{section:Deadlockerkennung allgemein} beschrieben. Anschließend
werden in \cref{section:Evaluierung von Methoden zur dynamischen
Deadlockerkennung} zwei Verfahren evaluiert. Betrachtet werden Programme die mit
der Echtzeit-Programmiersprache PEARL\autocite{PEARL} geschrieben wurden.

\section{Trace Funktion}\label{Validierung:Trace Funktion}
\begin{itemize}
  \item Beispielprogramme in PEARL definieren
  \item Anforderungen validieren:
  \begin{itemize}
    \item Laufzeiten der einzelnen Programme mit und ohne Trace Funktionalität
    bestimmen und vergleichen, mit Python Programm \cref{lst:Python_Benchmark_CPU}
    \item Speicherauslastung der einzelnen Programme mit und ohne Trace
    Funktionalität bestimmen und vergleichen, mit Python Programm \cref{lst:Python_Benchmark_Memory}
  \end{itemize}
\end{itemize}

\begin{listing}[ht]
  \inputminted[frame=lines,linenos]{python}{./Python/benchmark_cpu.py}
  \caption{Pythonskipt zur Messung der Laufzeit}
  \label{lst:Python_Benchmark_CPU}   
\end{listing} 

\begin{listing}[ht]
  \inputminted[frame=lines,linenos]{python}{./Python/benchmark_memory.py}
  \caption{Pythonskipt zur Messung der Speicherauslastung}
  \label{lst:Python_Benchmark_Memory}   
\end{listing} 

\section{Analyse Programm}
\begin{itemize}
  \item Die aus \cref{Validierung:Trace Funktion} erstellten Trace-Dateien mit
  dem erstellten Analyse Programm auswerten
\end{itemize}

\section{Visualisierung von potenziellen Deadlocks}
\begin{itemize}
  \item Die aus \cref{Validierung:Trace Funktion} erstellten Trace-Dateien
  auswerten und die potenziellen Deadlocks visuell als gerichtete Graphen
  aufzeigen
\end{itemize}